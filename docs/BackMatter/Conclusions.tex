\begin{conclusions}

    Los resultados presentados en este trabajo permiten afirmar que se han cumplido los objetivos específicos y se ha corroborado la hipótesis de trabajo formulada en la introducción. Efectivamente, el uso de un modelo de aprendizaje automático basado en \textit{transformers} permite capturar las complejas relaciones espaciotemporales que existen entre las ubicaciones de una trayectoria.
    
    La aplicación de TrajBERT para el completamiento de trayectorias implícitas a partir de datos de telefonía móvil en Cuba demuestra la viabilidad de un enfoque avanzado en un entorno con datos bastante dispersos. Se evidenció que su innovadora arquitectura, que incorpora un refinamiento espaciotemporal y la función STAL, realmente ofrece mejoras significativas en el rendimiento del modelo, superando los números alcanzados por el enfoque previo cubano con una arquitectura más simple basado en redes neuronales artificiales.
    
    Al comparar los experimentos realizados con los conjuntos de datos de ETECSA y \textit{HuMob}, se observa que en el contexto cubano el rendimiento del modelo es inferior, lo que apunta a una limitación atribuible a la calidad del conjunto de datos y no al método en sí. Esto sugiere que, si se contara con datos de entrenamiento de mayor precisión y densidad, por ejemplo, obtenidos mediante dispositivos GPS, el modelo podría alcanzar niveles de eficiencia en Cuba aún mayores, incluso en escenarios con datos muy dispersos.
    
    Los alcances prácticos de esta investigación son evidentes. Las aplicaciones potenciales van desde la modelación de la propagación de enfermedades basada en patrones de movilidad inferidos hasta la selección óptima de ubicaciones para la construcción de tiendas u hospitales. Otra propuesta interesante es la compresión y descompresión de datos de movilidad, es decir, representar trayectorias completas mediante un subconjunto crítico de puntos, lo que reduciría el volumen de almacenamiento requerido sin perder la utilidad analítica. En este sentido, TrajBERT podría desempeñar un papel clave al ``descomprimir'' estos datos, permitiendo obtener de vuelta la información detallada a partir de representaciones comprimidas.
    
    El análisis desarrollado en este trabajo, junto con la difusión de la base de código a la comunidad científica cubana, abre nuevas líneas de investigación para optimizar la aplicación de técnicas de aprendizaje profundo en el estudio de la movilidad humana en Cuba. Estos avances permitirán una comprensión más precisa de los patrones de movilidad y favorecerán el diseño de estrategias innovadoras en ámbitos tan relevantes como la planificación urbana, el transporte, los estudios epidemiológicos y la toma de decisiones en políticas públicas.
        
\end{conclusions}