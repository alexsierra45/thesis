\begin{recomendations}

    Finalmente, se proponen las siguientes recomendaciones para futuros trabajos, que permitirán ampliar el alcance de este estudio:
    
    \begin{itemize}
        \item Estandarizar los criterios de evaluación en modelos de movilidad en Cuba, definiendo métricas universales o protocolos de filtrado de datos que faciliten comparaciones objetivas entre enfoques.
        \item Evaluar la posibilidad de entrenar a TrajBERT con una base de datos cubana, quizá más reducida en tamaño pero compuesta por trayectorias mucho más densas y precisas, para determinar si ello mejora su rendimiento en trayectorias de baja densidad.
        \item Incorporar a la arquitectura de TrajBERT otros tipos de información contextual para que no se limite únicamente al análisis de trayectorias implícitas
        \item Investigar y diseñar modelos especializados que se adapten a contextos particulares, considerando tal vez las diferencias entre municipios, las variaciones climáticas y las características propias de cada período del año.
        \item Analizar la utilidad de los \textit{embeddings} de las zonas de transporte generados por TrajBERT durante el entrenamiento, para determinar si pueden aportar información complementaria en tareas de análisis de movilidad.
    \end{itemize}
    
\end{recomendations}