\begin{resumen}

	Este trabajo evalúa la aplicabilidad de TrajBERT para completar trayectorias a partir de datos de telefonía móvil en Cuba, enfrentando el reto de reconstruir secuencias fragmentadas y escasas. Basado en la arquitectura \textit{transformers}, TrajBERT modela patrones de movilidad bidireccionalmente, mejorando la predicción de ubicaciones faltantes gracias al aprovechamiento de las dependencias espaciotemporales entre las ubicaciones de una trayectoria. Se evaluó el modelo con datos del \textit{HuMob Challenge} 2023 y registros reales de La Habana, comparándolo con métodos tradicionales y estudios cubanos previos, evidenciando mayor efectividad en su desempeño. El estudio ofrece una herramienta valiosa para la planificación urbana, el transporte, los estudios epidemiológicos y la toma de decisiones en políticas públicas, aprovechando el uso de datos propiamente cubanos.
	
	\end{resumen}
	
	\begin{abstract}
	
	This work evaluates the applicability of TrajBERT for completing trajectories from mobile phone data in Cuba, addressing the challenge of reconstructing fragmented and scarce sequences. Based on the transformers architecture, TrajBERT models mobility patterns bidirectionally, enhancing the prediction of missing locations by leveraging the spatiotemporal dependencies among the points of a trajectory. The model was evaluated using data from the HuMob Challenge 2023 and real records from Havana, and its performance was compared with traditional methods and previous Cuban studies, demonstrating superior effectiveness. This study provides a valuable tool for urban planning, transportation, epidemiological studies, and public policy decision-making by making optimal use of uniquely Cuban data.
	
\end{abstract}