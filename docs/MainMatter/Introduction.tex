\chapter*{Introducción}\label{chapter:introduction}
\addcontentsline{toc}{chapter}{Introducción}

En la era del big data, la capacidad de analizar e interpretar datos complejos de trayectorias humanas se ha convertido en un 
elemento esencial para comprender y gestionar dinámicas sociales, urbanas y ambientales. Este análisis, impulsado por datos 
recolectados a través de la telefonía móvil, no solo permite identificar patrones de movilidad humana, sino que también informa 
la planificación urbana, optimiza el transporte y guía la formulación de políticas públicas. Sin embargo, estos datos suelen ser 
incompletos o dispersos debido a factores como la pérdida de señal o limitaciones en las capacidades de seguimiento. Por lo tanto, 
abordar los desafíos relacionados con su completamiento es crucial.

Dentro de este panorama, el completamiento de trayectorias humanas a partir de datos escasos representa un desafío significativo 
en nuestro país. En contextos de desarrollo como Cuba, donde la digitalización está en crecimiento pero los recursos y datos 
disponibles son limitados, los datos recolectados mediante la telefonía móvil ofrecen una oportunidad invaluable. Sin embargo, 
la dispersión, incompletitud e irregularidad en el tiempo de los datos, debido a apagones y averías en el sistema de 
telecomunicaciones, dificultan en gran medida este proceso. Por tal motivo se hace necesario desarrollar soluciones adaptadas a 
nuestras particularidades.

En los últimos años, diversas investigaciones han explorado métodos para el completamiento y la predicción de trayectorias, 
abarcando desde técnicas clásicas de interpolación hasta enfoques más recientes de aprendizaje profundo. En esto ha tenido un 
gran impacto la competencia HuMob (Human Mobility) Challenge que se propone anualmente romper el estado del arte de los modelos 
computacionales para la predicción de patrones de movilidad humana. Sin embargo, la mayoría de los enfoques están casi siempre 
orientados a entornos con abundancia de datos, dejando una brecha en la literatura sobre contextos con datos limitados. Destacan 
entonces enfoques como el de GRFTrajRec con una representación de trayectorias basada en grafos, la cual mejora la comprensión 
de las interacciones entre trayectorias y redes viales; gracias a un modelo seq2seq basado en intervalos espaciotemporales. Por 
su parte, los datos de registros de llamadas (CDR) en conjunto con un punto de vista basado en factorización tensorial permite 
reconstruir trayectorias individuales con alta precisión, incluso con solo el 1$\%$ de los datos originales. El uso de datos de 
telefonía en Cuba ha permitido analizar patrones de movilidad en diversas situaciones, como durante la pandemia de COVID-19, y 
ha sido fundamental en la integración de modelos para la propagación de epidemias. De igual forma, estudios han aprovechado estos 
datos para actualizar áreas de ubicación y comprender los flujos de movilidad entre zonas de transporte, lo que contribuye 
significativamente a la toma de decisiones y al control de crisis.

A diferencia de las investigaciones anteriores este trabajo se centra en evaluar la aplicabilidad y el desempeño de TrajBERT, 
un modelo basado en la arquitectura de transformer que permite aprovechar la información espacial y temporal. Este estudio se 
enfoca en adaptar el modelo a las particularidades locales, al emplear datos cubanos recolectados mediante telefonía móvil. 
Finalmente, se busca identificar tanto las fortalezas como las limitaciones del modelo, contribuyendo al avance del conocimiento 
en el campo.

\textbf{Obejtivo General}: Investigar la aplicabilidad y adaptabilidad del modelo TrajBERT para el completamiento de trayectorias 
humanas en el contexto cubano, utilizando datos recolectados mediante telefonía móvil, evaluando su desempeño en comparación con 
métodos tradicionales.

\textbf{Obejtivos Específicos}:

\begin{enumerate}
    \item Analizar las fortalezas y debilidades del modelo TrajBERT en el contexto del completamiento de trayectorias, evaluando 
    su arquitectura y capacidades técnicas.
    \item Implementar TrajBERT utilizando una base de datos de trayectorias cubanas recopiladas a través de la telefonía móvil 
    para evaluar su desempeño en la recuperación de trayectorias individuales.
    \item Comparar los resultados obtenidos por TrajBERT con los de métodos tradicionales y soluciones existentes en el campo, 
    identificando las diferencias en precisión y efectividad.
\end{enumerate}