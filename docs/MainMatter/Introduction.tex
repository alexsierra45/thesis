\chapter*{Introducción}\label{chapter:introduction}
\addcontentsline{toc}{chapter}{Introducción}

En la era de los grandes volúmenes de datos, analizar e interpretar trayectorias humanas complejas resulta esencial para comprender y gestionar dinámicas sociales, urbanas y ambientales. Este análisis no solo permite identificar patrones de movilidad humana, sino que también informa la planificación urbana, optimiza el transporte y guía la formulación de políticas públicas.

Los datos recolectados a través de la telefonía móvil desempeñan un papel fundamental en esta tarea, ya que ofrecen una fuente valiosa para capturar la movilidad humana. Sin embargo, suelen presentar problemas de incompletitud o dispersión debido a factores como la pérdida de señal o limitaciones en las capacidades de seguimiento. Esto hace imprescindible desarrollar estrategias eficaces para el completamiento de trayectorias.

En contextos de desarrollo como Cuba, donde la digitalización avanza pero los recursos y datos disponibles son limitados, la telefonía móvil representa una oportunidad invaluable para recolectar información sobre movilidad. No obstante, los registros obtenidos en el país no están exentos de las irregularidades antes mencionadas. Interrupciones en el suministro eléctrico y fallos en los sistemas de comunicación complican significativamente la recolección de datos. 

En este panorama, el completamiento de trayectorias humanas a partir de información incompleta se convierte en un desafío crucial, especialmente considerando que los datos provenientes de registros de telefonía móvil cubanos son altamente imprecisos. Esta imprecisión se debe a que la posición está inferida a partir de la localización de las torres telefónicas y su proyección sobre zonas de transporte, lo que introduce incertidumbres espaciales que pueden ser de cientos de metros o incluso kilómetros, además de una baja granularidad temporal, con registros separados por decenas de minutos o incluso horas. De ahí la necesidad de desarrollar soluciones adaptadas a las particularidades de Cuba.

Los enfoques tradicionales para el completamiento de trayectorias humanas han estado basados principalmente en técnicas como la interpolación lineal \cite{hoteit2014estimating}, modelos basados en asiganción de caminos más cortos \cite{st2014reconstructing}, el agrupamiento de trayectorias \cite{partsinevelos2005reconstructing} y el uso de factorización de matrices o tensores \cite{chen2019complete}. Estas técnicas, aunque son útiles en ciertos contextos, presentan limitaciones significativas. Por ejemplo, la interpolación lineal, aunque computacionalmente eficiente, no captura patrones complejos de movilidad; los modelos históricos y los métodos de agrupamiento tienden a fallar en entornos con alta variabilidad o datos limitados; y la factorización de matrices, si bien es efectiva para manejar datos dispersos como los registros de llamadas, no logra modelar adecuadamente las dinámicas temporales complejas.

Por otro lado, el análisis de trayectorias humanas como secuencias espaciotemporales resulta especialmente prometedor para enfoques modernos como los basados en arquitecturas de \textit{transformers} \cite{vaswani2017attention}. Estas arquitecturas destacan por su capacidad para modelar secuencias de manera eficiente, identificando relaciones clave entre sus elementos. En el caso particular de las trayectorias, este enfoque pudiera aprovechar la estructura temporal y espacial de los datos para extraer patrones relevantes, ofreciendo una ventaja significativa frente a métodos tradicionales que no consideran estas interdependencias de forma explícita. No obstante, aplicar estos modelos innovadores en contextos específicos, como el caso cubano, exige analizar minuciosamente las particularidades de los datos disponibles.

En Cuba, la telefonía móvil, gestionada por la Empresa de Telecomunicaciones de Cuba (ETECSA), ha generado un conjunto de registros espaciotemporales que, desde 2020, se emplean en estudios de movilidad realizados por el Centro de Sistemas Complejos de la Facultad de Física de la Universidad de La Habana. Estos datos, aunque menos precisos y abundantes que los que poseen empresas internacionales como Google, han demostrado su utilidad en contextos críticos, como la gestión de la pandemia de SARS-CoV2. Esto plantea el desafío de evaluar cómo el empleo de técnicas avanzadas pueden maximizar el valor informacional de estos datos locales, sentando las bases para abordar problemas específicos de movilidad y toma de decisiones en el país.

A partir de lo valorado anteriormente, podemos identificar como \textbf{problema científico} la imposibilidad, en el contexto cubano, de aplicar los métodos tradicionales para el completamiento con precisión de trayectorias humanas,. La \textbf{hipótesis} de trabajo es que el uso de un modelo basado en \textit{transformers} permitirá superar estas limitaciones al capturar de manera más efectiva las relaciones espaciotemporales presentes en los datos y proporcionar un completamiento de trayectorias más preciso.

Por tanto, y teniendo en cuenta los datos disponibles en el Centro de Sistemas Complejos y las condiciones establecidas con la empresa de telecomunicaciones para su uso, esta investigación aborda el problema de completar trayectorias de movilidad humana a partir de registros de telefonía móvil en Cuba. En particular, evalúa la aplicabilidad y desempeño de TrajBERT \cite{si2023trajbert}, un modelo basado en la arquitectura de \textit{transformers}. Este enfoque busca avanzar en el desarrollo de modelos de movilidad más precisos, explorando tanto sus fortalezas como limitaciones en entornos similares.

En este marco, el objetivo general de esta investigación es estudiar la aplicabilidad y adaptabilidad de TrajBERT para el completamiento de trayectorias humanas en el contexto cubano, utilizando datos de telefonía móvil recolectados en el país.

Para lograr este fin, se proponen los siguientes objetivos específicos:

\begin{enumerate} 
    \item Investigar el uso de datos de telefonía móvil en estudios de movilidad humana a nivel global, con énfasis en Cuba. 
    \item Revisar y analizar la literatura científica relacionada con los estudios de movilidad humana y las tecnologías empleadas en este campo. 
    \item Analizar las fortalezas y debilidades del modelo TrajBERT para el completamiento de trayectorias en cuanto a su arquitectura y capacidades técnicas. 
    \item Comparar los resultados obtenidos por TrajBERT en datos reales con los de métodos tradicionales. 
    \item Entrenar TrajBERT sobre una base de datos de trayectorias cubanas y comparar su desempeño con resultados previos.
\end{enumerate}

Esta tesis se organiza en tres capítulos. El \textbf{Capítulo 1} profundiza en la revisión de la literatura, haciendo énfasis en los métodos tecnológicos utilizados en el completamiento de trayectorias, mientras se analiza el uso de datos de telefonía móvil para estudiar patrones de movilidad humana, abordando desafíos técnicos y éticos, con particular interés en el caso cubano. En el \textbf{Capítulo 2} se analiza e implementa el modelo TrajBERT, evaluando sus fortalezas y debilidades mediante el entrenamiento con los datos de la competencia \textit{HuMob Challenge} 2023. Finalmente, en el \textbf{Capítulo 3} se entrena y evalúa TrajBERT con datos de trayectorias cubanas para contextualizar su desempeño y compararlo con otro método existente en Cuba.
