\chapter*{Introducción}\label{chapter:introduction}
\addcontentsline{toc}{chapter}{Introducción}

En la era del \textit{big data}, la capacidad de analizar e interpretar datos complejos de trayectorias humanas se ha convertido en un elemento esencial para comprender y gestionar dinámicas sociales, urbanas y ambientales. Este análisis, impulsado por datos recolectados a través de la telefonía móvil, no solo permite identificar patrones de movilidad humana, sino que también informa la planificación urbana, optimiza el transporte y guía la formulación de políticas públicas. Sin embargo, estos datos suelen ser incompletos o dispersos debido a factores como la pérdida de señal o limitaciones en las capacidades de seguimiento, lo que hace imprescindible desarrollar estrategias eficaces para su completamiento.

En contextos de desarrollo como Cuba, donde la digitalización está en crecimiento, pero los recursos y datos disponibles son limitados, los datos recolectados mediante la telefonía móvil ofrecen una oportunidad invaluable. No obstante, la dispersión, incompletitud e irregularidad en el tiempo de los datos, debido a apagones y averías en el sistema de telecomunicaciones, dificultan en gran medida este proceso. Dentro de este panorama, el completamiento de trayectorias humanas a partir de datos escasos representa un reto significativo en nuestro país, de ahí la necesidad de desarrollar soluciones adaptadas a nuestras particularidades.

Los enfoques tradicionales para el completamiento de trayectorias humanas han estado basados principalmente en técnicas como la interpolación lineal, los modelos basados en patrones históricos, el \textit{clustering} de trayectorias y el uso de factorización de matrices o tensores. Estas técnicas, aunque útiles en ciertos contextos, presentan limitaciones significativas. Por ejemplo, la interpolación lineal, aunque computacionalmente eficiente, no captura patrones complejos de movilidad; los modelos históricos y los métodos de \textit{clustering} tienden a fallar en entornos con alta variabilidad o datos limitados; y la factorización de matrices, si bien efectiva para manejar datos dispersos como los registros de llamadas, no logra modelar adecuadamente las dinámicas temporales complejas.

Por otro lado, el análisis de trayectorias humanas como secuencias espaciotemporales resulta especialmente prometedor para enfoques modernos como los basados en arquitecturas de \textit{transformers} \cite{vaswani2017attention}. Estas arquitecturas destacan por su capacidad para modelar secuencias de manera eficiente, identificando relaciones clave entre los puntos de la trayectoria. Este enfoque permite aprovechar la estructura temporal y espacial de los datos para extraer patrones relevantes, ofreciendo una ventaja significativa frente a métodos tradicionales que no consideran estas interdependencias de forma explícita.

Aunque las arquitecturas basadas en \textit{transformers} han mostrado gran capacidad para analizar secuencias complejas, su aplicación en contextos específicos, como el caso cubano, requiere una exploración cuidadosa de las características particulares de los datos disponibles. En Cuba, la telefonía móvil, a través de la Empresa de Telecomunicaciones de Cuba (Etecsa), ha generado un conjunto de registros espaciotemporales que se emplean en estudios de movilidad desde 2020 por el Centro de Sistemas Complejos en la Facultad de Física de la Universidad de la Habana. Estos datos, aunque menos precisos y abundantes que los proporcionados por empresas internacionales como Google, han demostrado su utilidad en contextos críticos, como la gestión de la pandemia de SARS-CoV2. Esto plantea el desafío de evaluar cómo técnicas avanzadas pueden maximizar el valor informacional de estos datos locales, sentando las bases para abordar problemas específicos de movilidad y toma de decisiones en el país.

A partir de lo discutido anteriormente, podemos identificar como \textbf{problema científico} que los métodos tradicionales no son capaces de completar trayectorias humanas con precisión en escenarios de datos dispersos e irregulares, como ocurre en Cuba. La \textbf{hipótesis} de trabajo es que el uso de un modelo basado en \textit{transformers} permitirá superar estas limitaciones al capturar de manera más efectiva las relaciones espaciotemporales presentes en los datos y proporcionar un completamiento de trayectorias más preciso.

Por tanto, y teniendo en cuenta los datos disponibles en el Centro de Sistemas Complejos y las condiciones establecidas con la empresa de telecomunicaciones para su uso, esta investigación aborda el problema de completar trayectorias de movilidad humana a partir de registros de telefonía móvil en Cuba. A diferencia de trabajos previos, se evalúa la aplicabilidad y desempeño de TrajBERT \cite{si2023trajbert}, un modelo basado en la arquitectura de \textit{transformers} , adaptado a las particularidades locales. Este enfoque busca avanzar en el desarrollo de modelos de movilidad más precisos, explorando tanto sus fortalezas como sus limitaciones en entornos similares.

En este marco, el objetivo general del estudio es investigar la aplicabilidad y adaptabilidad de TrajBERT para el completamiento de trayectorias humanas en el contexto cubano, utilizando datos recolectados mediante telefonía móvil.

 Para lograr este fin, se proponen los siguientes objetivos específicos:

 \begin{enumerate}
    \item Explorar el uso de datos de telefonía móvil para estudios de movilidad humana a nivel global, específicamente en Cuba, e investigar la literatura relacionada con el tema y las herramientas tecnológicas utilizadas, como el aprendizaje profundo y las arquitecturas basadas en \textit{transformer}.
    \item Analizar las fortalezas y debilidades del modelo TrajBERT en el contexto del completamiento de trayectorias, mediante la evaluación de su arquitectura y capacidades técnicas.
    \item Implementar TrajBERT utilizando una base de datos de trayectorias cubanas para evaluar su desempeño en la recuperación de trayectorias individuales.
    \item Comparar los resultados obtenidos por TrajBERT con los de métodos tradicionales y soluciones existentes en el campo, en cuanto a las diferencias de precisión y efectividad.
\end{enumerate}

La estructura de la tesis está diseñada para cumplir con los objetivos planteados mediante tres capítulos interrelacionados.

El Capítulo 1 se centra en la revisión de la literatura y las herramientas tecnológicas utilizadas, abordando el objetivo de investigar las bases teóricas y metodológicas del tema. Aquí se analizan los avances recientes en completamiento de trayectorias, destacando tanto métodos tradicionales como modernos, además de explorar en profundidad las arquitecturas basadas en \textit{transformer} y sus aplicaciones en el aprendizaje profundo. Esto establece un marco teórico sólido que respalda el desarrollo posterior de la tesis.

El Capítulo 2 aborda el análisis y la implementación del modelo TrajBERT, cumpliendo con los objetivos relacionados con la evaluación de sus fortalezas y debilidades, así como su implementación práctica. Posteriormente, se implementa el modelo utilizando la base de datos pública de la competencia HuMob (Human Mobility) Challenge 2023, lo que incluye el preprocesamiento de datos, entrenamiento del modelo y evaluación de su desempeño en la recuperación de trayectorias individuales mediante métricas específicas.

Finalmente, el Capítulo 3 se enfoca en la implementación y evaluación del modelo TrajBERT utilizando datos de trayectorias cubanas, permitiendo cumplir con el objetivo de contextualizar su desempeño en un entorno local y comparar sus resultados con métodos tradicionales y soluciones existentes en el campo. Este capítulo concluye con un análisis sobre las ventajas y limitaciones del modelo TrajBERT cuando se aplica a datos locales, ofreciendo recomendaciones para futuras investigaciones en el área.

Esta estructura garantiza que cada objetivo se aborde de manera sistemática y que los capítulos estén interconectados de forma coherente para proporcionar resultados concluyentes.